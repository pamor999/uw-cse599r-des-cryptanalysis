\documentclass[9pt]{beamer}

\usepackage{amsmath,amssymb,amstext,amsfonts,setspace,dsfont,amsmath,amsthm,array,nomencl,hyperref,pgf,xcolor}

\usepackage{beamerthemesplit}

%\usetheme{Dresden}
\usetheme{Ilmenau}

\setbeamertemplate{navigation symbols}{} %no navigation symbols

\setbeamertemplate{footline}
{%
\leavevmode%
\hbox{%

\begin{beamercolorbox}[wd=.3\paperwidth,ht=2.5ex,dp=1.125ex,left]{author
in head/foot}%
\usebeamerfont{author in head/foot}\hspace{.3cm}\insertshortauthor\hspace{.3cm}
\end{beamercolorbox}%

\begin{beamercolorbox}[wd=.5\paperwidth,ht=2.5ex,dp=1.125ex,center]{title
in head/foot}%
\usebeamerfont{title in head/foot}\hspace{.3cm}\insertshorttitle
\end{beamercolorbox}%

\begin{beamercolorbox}[wd=.2\paperwidth,ht=2.5ex,dp=1.125ex,right]{author
in head/foot}%
\usebeamerfont{author in head/foot}Slide\hspace{.1cm}\insertframenumber\hspace{.1cm}of\hspace{.1cm}\inserttotalframenumber\hspace{.3cm}
\end{beamercolorbox}%
}%
\vskip0pt%
}

\setbeamertemplate{blocks}[rounded][shadow=true]

\title{Linear and Differential Cryptanalysis of DES}
\author{Slava Chernyak and Sourav Sen Gupta}
\institute{University of Washington}
\date{February 2, 2009}

\begin{document}

\begin{frame}
\titlepage
\end{frame}

\section{Introduction}
\begin{frame}{Introduction}
\begin{block}{Quack!}
Introduction to DES and the two attacks
\end{block}
\end{frame}

\subsection{Data Encrytption Standard}
\begin{frame}
DES structure

\end{frame}

\begin{frame}
DES pieces

\end{frame}

\begin{frame}
DES mathematics

\end{frame}

\subsection{Linear Cryptanalysis}
\begin{frame}
Vulnerabilities of DES

\end{frame}

\begin{frame}
How to exploit those - Linear attack idea

\end{frame}


\subsection{Differential Cryptanalysis}
\begin{frame}
Vulnerabilities of DES

\end{frame}

\begin{frame}
How to exploit those - Differential attack idea

\end{frame}

\section{Mathematical Framework}
\begin{frame}
Mathematical description of the attacks

\end{frame}

\subsection{Simplification WLOG}
\begin{frame}
Reducing DES down to SPN

\end{frame}

\begin{frame}
Introducing SPN mathematically

\end{frame}

\subsection{Linear Attack on SPN}
\begin{frame}
Basic math concepts behind Linear attack ...

\end{frame}

\begin{frame}
Intro to linear / affine approximation ... 

\end{frame}

\begin{frame}
Linear characteristic / approximation of SBox ...

\end{frame}

\begin{frame}
Probabilities for the approximations ...

\end{frame}

\begin{frame}
Piling Up lemma ...

\end{frame}

\begin{frame}
Combining across SPN rounds ... 

\end{frame}

\begin{frame}
The algorithm for the attack ... 

\end{frame}

\begin{frame}
Probabilistic justification ...

\end{frame}

\subsection{Differential Attack on SPN}
\begin{frame}
Basic math concepts behind Differential attack ...

\end{frame}

\begin{frame}
Intro to differentials ... 

\end{frame}

\begin{frame}
Differential characteristic / approximation of SBox ...

\end{frame}

\begin{frame}
Probabilities for the approximations ...

\end{frame}

\begin{frame}
Analogue of Piling Up lemma ...

\end{frame}

\begin{frame}
Combining across SPN rounds ... 

\end{frame}

\begin{frame}
The algorithm for the attack ... 

\end{frame}

\begin{frame}
Probabilistic justification ...

\end{frame}

\subsection{Actual Attacks}
\begin{frame}
Demo of attacks on SPN ... Runtime comparisons with Brute Force ...

\end{frame}

\begin{frame}
Break ... After coming back from the break, we should refresh the basic ideas behind the attacks once again ...

\end{frame}


\section{Attacks on DES}
\begin{frame}
Back to DES ... Pull back from SPN ...

\end{frame}

\subsection{Linear Attack: 3-round DES}
\begin{frame}
Linear Attack idea ... 

\end{frame}

\begin{frame}
Matsui NS for SBoxes ...

\end{frame}

\begin{frame}
Linear approximations for SBoxes ...

\end{frame}

\begin{frame}
Piling up over 3 rounds ...

\end{frame}

\begin{frame}
Demo of the attack ...

\end{frame}

\subsection{Differential Attack: 3-round DES}
\begin{frame}
Differential Attack idea ... 

\end{frame}

\begin{frame}
Differentials for SBoxes ...

\end{frame}

\begin{frame}
Differential characteristics for SBoxes ...

\end{frame}

\begin{frame}
Piling up over 3 rounds ...

\end{frame}

\begin{frame}
Demo of the attack ...

\end{frame}

\subsection{Extension to full DES}
\begin{frame}
Extension of the same ideas to 6 / 8 / 16 round DES ... Probabilistic explanation ...

\end{frame}


\section{Conclusion}
\begin{frame}
Concluding remarks about the project ...

\end{frame}

\subsection{Independence of S-Boxes}
\begin{frame}
Refresh math idea ... Independence assumption ... 

\end{frame}

\begin{frame}
What if that is false ? ... Discussion ...

\end{frame}

\subsection{Number of Pairs Required}
\begin{frame}
Probabilistic possibility of the attacks ... Pairs of data points needed ... 

\end{frame}

\begin{frame}
Generalized attack idea on any given system ... Discussion ...

\end{frame}

\subsection{Links between the Attacks}
\begin{frame}
Links between Linear and Differential Attacks ... 

\end{frame}

\begin{frame}
Mathematically and while implementing ...

\end{frame}

\subsection{Take home Lesson}
\begin{frame}
If you are mounting the attacks ... To Do / Not To Do ...

\end{frame}

\begin{frame}
If you are building a cryptosystem ... To Do / Not To Do ...

\end{frame}

\begin{frame}
Questions ... Doubts ... Coding ... Misc ... Discussion ...

\end{frame}

\end{document}
